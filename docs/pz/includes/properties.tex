\section{Технические характеристики}

\subsection{Назначение программы}

\subsubsection{Функциональное назначение программы}

Функциональным назначением программы является сводный и реляционный анализ данных, экспортируемых из аккаунтов Telegram \cite{telegram}.

\subsubsection{Эксплуатационное назначение программы}

Приложение позволяет находить связи между пользователями в выгрузке сообщений из Telegram.
После импорта .json файла с данными, пользователь приложения может получить для заданного списка пользователей Telegram граф их связей.
Под связями подразумеваются как прямые (A писал B), так и косвенные (A и B писали в один чат, A упоминал B в сообщении, А писал в чат в котором упоминался B).

\subsection{Алгоритм работы программы}

Перед проведением анализа пользователь выбирает экспортированный из Telegram .json файл на своем устройстве. Этот файл передается в веб-приложение.
При этом файл не загружается на удаленный сервер - вся обработка производится на устройстве. 
Это предоставляет приложению следующие преимущества:

\begin{itemize}
    \item Анализ и визуализация данных проходят максимально быстро - нет необходимости загружать наборы данных, достигающие гигабайт, на удаленный сервер, и загружать с него сгенерированные графы.
    \item Приложение максимально устойчиво к нагрузке. На серверную часть нагрузка сводится к обеспечению доступа к нескольким статическим файлам, что обеспечивает устойчивость работы даже на сравнительно маломощном оборудовании.
    \item Данные не покидают устройство пользователя. Это гарантирует анонимность, приватность и предотвращает компрометацию аккаунта.
\end{itemize}

После загрузки вычисляется граф зависимостей, имеющий следующую структуру:

\begin{lstlisting}
class Graph {
    chatIdToName = {};
    chatNameToId = {};

    actorIdToName = {};
    actorNameToId = {};

    chatIdToMessages = {}; // Messages sent
    actorIdToMessages = {};

    actorToChatMapping = {}; // User writing something in chat. [user_id][chat_id] = [messages]
    actorToActorMapping = {}; // User mentioning another user. [user_a_id][user_b_id] = [messages]

    chatCategories = {};
}
\end{lstlisting}

Собираются отношения Telegram ID к видимым именам для всех объектов (пользователи, чаты, каналы).
Они используются для упрощения взаимодействия с пользователем (автодополнение, имена в графах вместо ID).

Кроме того, сообщения группируются по чатам и по пользователям. 
На этом этапе строится базовая связь чатов с пользователями (пользователь писал в чат) и пользователей с пользователями (пользователь упоминал пользователя в одном из своих сообщений).
Стоит заметить, что построение связей упоминаний пользователей в может найти интересные связи, но при этом приводит к созданию в графе несуществующих связей.
Например, при тестировании обнаружилось, что очень много пользователей связаны с одним конкретным пользователем. Причина была в том, что имя пользователя - ``Да'', что, конечно, много где упоминалось в переписке.

Данные вычисляются один раз, после чего построение графа использует заранее просчитанный набор связей.
Это позволяет значительно уменьшить задержку при взаимодействии с пользователем и сделать скорость построения графа незначительной.
Так, даже если время импорта файла будет порядка 20-30 секунд, непосредственно построение графа всегда будет измеряться в миллисекундах.

При непосредственно построении графа мы уже имеем заранее просчитанную информацию о чатах каждого из пользователей, так что задача сводится к поиску общих в массивах.
В совокупности с использованием производительной библиотеки для визуализации данных (Apache ECharts \cite{echarts}) это позволяет приложению почти мгновенно рисовать графы любого размера. 

\subsection{Входные и выходные данные}

\subsubsection{Организация входных данных}

Входные данные - .json файл, экспортированный из приложения Telegram \cite{telegramExport}.

\subsubsection{Обоснование метода организации входных данных}

Для подобного анализа требуется, с одной стороны, проанализировать большой набор данных (речь может идти о миллионах сообщений),
с другой стороны, сделать это без потери доверия пользователя (если пользователь не доверяет сервису, он не будет загружать туда свою личную переписку).

Анализ экспорта позволяет нам получить все сообщения не взаимодействуя с Telegram напрямую (а такое большое количество запросов к API могло бы привести к блокировке), что делает сервис надежнее и быстрее.
Более того, разница по скорости - не просто вопрос нескольких секунд: при прямом взаимодействии с Telegram речь идет о замедлении как минимум на несколько порядков, т.к. Telegram ограничивает количество запросов, которые могут быть выполнены сторонними клиентами.

Кроме того, так пользователю не нужно давать доступ к своему Telegram-аккаунту (что, для большинства пользователей, было бы неприемлемо). 
В совокупности с обработкой данных исключительно на устройстве пользователя это обеспечивает максимальный уровень анонимности и приватности.

\subsubsection{Организация выходных данных}

Базовая статистика (количество пользователей в экспорте, контактов, базовая информация о аккаунте) выводится в текстовом виде.
Анализ отношений пользователей выводится на экран в виде интерактивного графа, отображающего пользователей и ассоциированные с ними чаты.
В качестве дополнительной информации на графе по разному (тип и цвет объекта) отображаются разные элементы графа: пользователи, личные чаты, публичные и приватные группы, каналы.
С каждым объектом указывается его уникальный Telegram ID.

\subsubsection{Обоснование метода организации выходных данных}

Базовая статистика - набор из нескольких чисел и имен, простое текстовое представление для нее ожидаемо, нет повода от него отказываться.

Представление связей между пользователями в виде интерактивного графа мотивированно доступностью пользователям: на маленьких примерах и статическая картинка, и интерактивный вариант, который можно подвигать или приблизить могут быть достаточно удобны, но на больших, запутанных схемах интерактивность значительно улучшает пользовательский опыт.
Цветовая маркировка различных элементов улучшает читаемость графа, особенно в случаях конфликта имен (например, так проще отличить пользователя ``Котенок'' и канал с картинками с аналогичным названием).
Указание Telegram ID решает случаи конфликтов, когда объекты одного типа имеют одинаковое имя (например, два разных пользователя, но оба с именем ``Олег'').