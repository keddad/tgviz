\subsection{Ориентировочная экономическая эффективность}

В рамках данной работы расчет экономической эффективности не предусмотрен.

\subsection{Предполагаемая потребность}

У активных пользователей Telegram могут быть тысячи контактов, которые пересекаются через разные чаты и каналы. Понять, откуда ты знаешь этого человека, или пересекались ли люди в такой ситуации - сложная, но регулярно необходимая задача.

Такой инструмент позволит строить потрет человека на основании его публичных сообщений (я вижу, что он активно обсуждал в чатах манулов), или оценивать его связи с другими людьми (я вижу, что и Алиса, и Боб регулярно пишут в чат по криптографии).

\subsection{Экономические преимущества разработки по сравнению с отечественными и зарубежными аналогами}

В свободном доступе существуют только инструменты для вычисления агрегированной статистики по экспорту данных. Ни один общедоступный (включая коммерческие) инструмент не подразумевает анализ связей отдельных пользователей, на который нацелено разрабатываемое приложение.