\section{Введение}

\subsection{Наименование программы}

\subsubsection{Наименование программы на русском языке}

Визуализатор социального графа <<Telegram>>.

\subsubsection{Наименование программы на английском языке}

<<Telegram>> social graph visualizer.

\subsection{Краткая характеристика области применения}

Для человека, активно пользующегося приложением <<Telegram>>, нормальной является ситуация, когда количество чатов и каналов достигает нескольких тысяч.
В такой ситуации ответить на вопросы ''Откуда я знаю этого человека?'', ''Что связывает этих двух людей?'' без средств автоматизации невозможно.

Встроенная в клиенты <<Telegram>> функциональность позволяет найти только базовое пересечение: список чатов, в которых находитесь вы и другой пользователь.
При этом поиск связей между другими пользователями невозможен, как и поиск по другим параметрам (''Оба пользователя писали в чат X'', ''Один из пользователей упоминал другого''), даже если пользователю доступна эта информация.

Цель данного приложения - обеспечить подобную функциональность анализа для пользователя, сохраняя его приватность (данные о переписках, необходимые для анализа, не должны покидать устройство пользователя).
При этом для пользователя такой анализ должен быть максимально абстрагирован от формата хранения данных, используемого <<Telegram>>, и прочих технических особенностей. Непосредственно поиск пересечений должен сводиться к указанию списка искомых контактов.