% creator: ilyakooo0
% extra credit: dashared
% version: 1.0.1

\PassOptionsToPackage{main=russian,english}{babel}

\documentclass{TechDoc}

\title{Визуализация социального графа пользователя Telegram}
\author{Студент группы БПИ 213}{Н. А. Бирюлин}
\academicTeacher{НИУ ВШЭ, Приглашенный Преподаватель}{А. А. Топтунов}

\documentTitle{Техническое задание}
\documentCode{RU.17701729.05.04-01 ТЗ 01-1}

\begin{document}
    \maketitle
    
    \tableofcontents
    
    \section{Введение}

\subsection{Наименование программы}

\subsubsection{Наименование программы на русском языке}

Визуализатор социального графа <<Telegram>>.

\subsubsection{Наименование программы на английском языке}

<<Telegram>> social graph visualizer.

\subsection{Краткая характеристика области применения}

Для человека, активно пользующегося приложением <<Telegram>>, нормальной является ситуация, когда количество чатов и каналов достигает нескольких тысяч.
В такой ситуации ответить на вопросы ''Откуда я знаю этого человека?'', ''Что связывает этих двух людей?'' без средств автоматизации невозможно.

Встроенная в клиенты <<Telegram>> функциональность позволяет найти только базовое пересечение: список чатов, в которых находитесь вы и другой пользователь.
При этом поиск связей между другими пользователями невозможен, как и поиск по другим параметрам (''Оба пользователя писали в чат X'', ''Один из пользователей упоминал другого''), даже если пользователю доступна эта информация.

Цель данного приложения - обеспечить подобную функциональность анализа для пользователя, сохраняя его приватность (данные о переписках, необходимые для анализа, не должны покидать устройство пользователя).
При этом для пользователя такой анализ должен быть максимально абстрагирован от формата хранения данных, используемого <<Telegram>>, и прочих технических особенностей. Непосредственно поиск пересечений должен сводиться к указанию списка искомых контактов.

    \section{Назначение программы}

\subsection{Функциональное назначение программы}

Функциональным назначением продукта является обеспечение процесса анализа данных, экспортированных из приложения <<Telegram>>.
Анализ заключается как в отображении сводной статистики по экспортированным данным, так и в поиске связей (общих чатов, каналов, упоминаний) между несколькими пользователями <<Telegram>> из информации, доступной в данных.

\subsection{Эксплуатационное назначение}

Приложение может быть использовано любым человеком для анализа связей в доступной ему переписке (в экспорте данных из аккаунта <<Telegram>>).
Пользователь, загрузив данные, может получить как общую информацию о переписках (количество, пользователи, чаты и каналы), так и информацию о общих связях отдельных пользователей.
Информация о связях предоставляется в формате интерактивного графа, вся обработка происходит в браузере, на устройстве пользователя.

\subsection{Состав функций}

Приложение должно: 

\begin{enumerate}
    \item импортировать данные экспорта сообщений из аккаунта Telegram;
    \item отображать сводную статистику по экспорту;
    \begin{enumerate}
        \item общее количество контактов и сообщений
        \item список известных чатов, каналов для указанного канала
    \end{enumerate}
    \item анализировать связи между выбранными контактами в экспорте;
    \begin{enumerate}
        \item список общих чатов, сообщения в этих чатах
        \item упоминания пользователями друг друга
        \item упоминания общих контактов (A упоминает B, C упоминает B)
        \item упоминания общих чатов/каналов (A, B упоминают один канал)
    \end{enumerate}
\end{enumerate}

    
    \section{Назначение разработки}

\subsection{Функциональное назначение}

Функциональным назначением продукта является обеспечение процесса анализа данных, экспортированных из приложения <<Telegram>>.
Анализ заключается как в отображении сводной статистики по экспортированным данным, так и в поиске связей (общих чатов, каналов, упоминаний) между несколькими пользователями <<Telegram>> из информации, доступной в данных.

\subsection{Эксплуатационное назначение}

Приложение может быть использовано любым человеком для анализа связей в доступной ему переписке (в экспорте данных из аккаунта <<Telegram>>).
Пользователь, загрузив данные, может получить как общую информацию о переписках (количество, пользователи, чаты и каналы), так и информацию о общих связях отдельных пользователей.
Информация о связях предоставляется в формате интерактивного графа, вся обработка происходит в браузере, на устройстве пользователя.

    \section{Требования к программе}
    
    \subsection{Требования к функциональным характеристикам}
    
    Приложение должно реализовывать функциональность анализа данных переписок пользователя, экспортированных из приложения <<Telegram>>. Сообщения должны загружаться через механизм экспорта данных из приложения.

После загрузки данных в приложение пользователю должны быть доступны как опции базового сводного анализа данных, так и поиска связей между отдельными пользователями.

Разрабатываемое приложение должно: 

\begin{enumerate}
    \item импортировать данные экспорта сообщений из аккаунта Telegram;
    \item отображать сводную статистику по экспорту;
    \begin{enumerate}
        \item общее количество контактов и сообщений
        \item список известных чатов, каналов для указанного канала
    \end{enumerate}
    \item анализировать связи между выбранными контактами в экспорте;
    \begin{enumerate}
        \item список общих чатов, сообщения в этих чатах
        \item упоминания пользователями друг друга
        \item упоминания общих контактов (A упоминает B, C упоминает B)
        \item упоминания общих чатов/каналов (A, B упоминают один канал)
    \end{enumerate}
\end{enumerate}


    \subsubsection{Организация входных данных}

    Задача приложения - обработка данных, экспортированных из <<Telegram>>.

    Такой экспорт выполняется из клиента <<Telegram>> для компьютеров. Экспорт должен выполняться в машинно-читаемом формате. Входными данными для анализа, загружаемыми в приложение, является .json файл в папке экспорта.

    \subsubsection{Организация выходных данных}
    
    Выходными данными являются агрегированные данные экспорта и интерактивные графы, доступные в приложении. Выгрузка данных из приложения не предусмотрена.
        
    \subsection{Требования к временным характеристикам}

    Первоначальный импорт данных не должен занимать больше, чем одну минуту. Все последующие операции анализа графа должны выполняться не более, чем за 20 секунд.

    \subsection{Требования к интерфейсу}
        
    Приложение должно иметь понятный пользователю интерфейс, позволяющей пользователю работать с программой с минимальной предварительной подготовкой. 

Веб-интерфейс должен позволять:
\begin{enumerate}
    \item загружать входные данные
    \item получать сводную статистику, сгенерированную на основании этих данных
    \item выбирать неограниченное количество контактов для анализа связей
    \item взаимодействовать с графом, полученным в результате анализа связей:
    \begin{enumerate}
        \item Для пользователей - получать информацию о телефонных номерах, где она доступна
        \item Для групп и каналов - получить информацию о известных сообщениях в них
        \item Для сообщений - получать информацию о времени отправки, авторе
    \end{enumerate}
\end{enumerate}

    
    \subsection{Условия эксплуатации}
    
    \subsubsection{Требования к пользователю}

Специальные требования к пользователям не предъявляются.

\subsection{Требования к составу и параметрам технических средств}

Для корректной работы пользователю требуется компьютер, характеристики которого соответствуют характеристикам ПО, описанным в разделе <<Требования к программным средствам, используемым программой>>.

Для серверной части требуется виртуальная машина как минимум с 8 GB оперативной памяти, 4 vCPU и 10 GB места на диске.
    
\subsection{Требования к исходным кодам и языкам программирования}

Требований к имплементации программного обеспечения не предъявляется.

\subsection{Требования к программным средствам, используемым программой}

На устройстве пользователя должен быть установлен один из совместимых браузеров:

\begin{enumerate}
    \item Firefox версии не менее 122.0
    \item Chrome версии не менее 121.0
    \item Яндекс.Браузер версии не менее 21.0
    \item Microsoft Edge версии не менее 121.0
\end{enumerate}

На сервере должен быть установлен Docker версии не менее 25.0 и docker-compose версии не менее 1.29.2.

\subsection{Требования к составу сетевых средств}

Как на устройстве пользователя, так и на сервере приложения должен быть обеспечен доступ в интернет.

\subsection{Требования к маркировке и упаковке}

Приложение размещается в виде веб-сайта, требований к маркировке и упаковке не предъявляется.

\subsection{Требования к транспортировке и хранению}

Исходный текст программы хранится в репозитории на платформе Github. Требований к транспортировке не предъявляется.


    \section{Требования к программной документации}

    \subsection{Предварительный состав программной документации}

\begin{enumerate}
    \item «Визуализация социального графа пользователя Telegram». Техническое задание (ГОСТ 19.201-78)\cite{gostTZ}.
    \item «Визуализация социального графа пользователя Telegram». Программа и методика испытаний (ГОСТ 19.301-78)\cite{gostPMI}.
    \item «Визуализация социального графа пользователя Telegram». Пояснительная записка (ГОСТ 19.404-79)\cite{gostPZ}.
    \item «Визуализация социального графа пользователя Telegram». Руководство оператора (ГОСТ 19.505-79)\cite{gostRO}.
    \item «Визуализация социального графа пользователя Telegram». Текст программы (ГОСТ 19.401-78)\cite{gostTP}.
\end{enumerate}

\subsection{Специальные требования к программной документации}

\begin{enumerate}
    \item Все документы к программе должны быть выполнены в соответствии с ГОСТ 19.106-78\cite{gostDoc} и ГОСТ к этому виду документа. 
    \item Пояснительная записка должна быть загружена в систему Антиплагиат через LMS «НИУ ВШЭ»
    \item Техническое задание, пояснительная записка, прочие документы должны быть подписаны исполнителем и руководителем разработки
    \item Документация и программа сдается в электронном виде в форматах .pdf, .docx, в архивах форматов .zip и .rar.
    \item За три дня до защиты комиссии все материалы курсового проекта должны быть загружены одним или несколькими архивами в проект дисциплины «Курсовой проект» в личном кабинете образовательной системы Smart LMS НИУ ВШЭ.
\end{enumerate}

    \section{Технико-экономические показатели}
    
    \section{Ожидаемые технико-экономические показатели}

\subsection{Экономическая эффективность}

Расчёт экономической эффективности в рамках работы не предусмотрен.

\subsection{Предполагаемая потребность}

Пупупу

\subsection{Преимущества разработки по сравнению с отечественными и зарубежными аналогами}

Пупупу

    \section{Стадии и этапы разработки}
    
    \begin{enumerate}
    \item техническое задание:
    \begin{enumerate}
        \item этапы разработки:
        \begin{enumerate}
            \item обоснование необходимости разработки программы; 
            \item постановка задачи; 
            \item сбор исходных материалов; 
            \item выбор и обоснование критериев эффективности и качества разрабатываемой программы; 
            \item обоснование необходимости проведения научно-исследовательских работ; 
        \end{enumerate}
        \item разработка и утверждение технического задания:
        \begin{enumerate}
            \item определение требований к программе; 
            \item определение стадий, этапов и сроков разработки программы и документации на неё; 
            \item согласование и утверждение технического задания; 
        \end{enumerate}
    \end{enumerate}
\item технический проект:
\begin{enumerate}
    \item разработка технического проекта:
    \begin{enumerate}
        \item уточнение структуры входных и выходных данных; 
        \item разработка алгоритма решения задачи; 
        \item определение формы представления входных и выходных данных; 
        \item разработка структуры программы; 
        \item окончательное определение конфигурации технических средств. 
    \end{enumerate}
    \item утверждение технического проекта:
    \begin{enumerate}
        \item разработка пояснительной записки; 
        \item согласование и утверждение технического проекта. 
    \end{enumerate}
\end{enumerate}
\item рабочий проект:
\begin{enumerate}
    \item разработка программы:
    \begin{enumerate}
        \item программирование и отладка программы. 
    \end{enumerate}
    \item разработка программной документации:
    \begin{enumerate}
        \item разработка программных документов в соответствии с требованиями ГОСТ 19.101-77\cite{gostAll}. 
    \end{enumerate}
    \item испытания программы:
    \begin{enumerate}
        \item разработка, согласование и утверждение порядка и методики испытаний; 
        \item корректировка программы и программной документации по результатам испытаний.
    \end{enumerate}
\end{enumerate}
\end{enumerate}

    \section{Сроки разработки и исполнители}
    
    Разработка должна завершиться к 25 марта 2024 года. Исполнитель - Бирюлин Н. А.

    \section{Порядок контроля и приемки}

    Контроль и приемка разработки осуществляются в соответствии с документом «Программа и методика испытаний» (ГОСТ 19.301-79)\cite{gostPMI}.
    
    \begin{thebibliography}{9}
        \bibitem{gostAll} ГОСТ 19.101-77. ЕСПД. Виды программ и программных документов. --- М.: ИПК Издательство стандартов, 2001.
        \bibitem{gostTZ} ГОСТ 19.201-78. ЕСПД. Техническое задание. Требования к содержанию и оформлению. --- М.: ИПК Издательство стандартов, 2001.
        \bibitem{gostPMI} ГОСТ 19.301-79. ЕСПД. Программа и методика испытаний. Требования к содержанию и оформлению. --- М.: ИПК Издательство стандартов, 2001.
        \bibitem{gostTP} ГОСТ 19.401-78. ЕСПД. Текст программы. Требования к содержанию и оформлению. --- М.: ИПК Издательство стандартов, 2001.
        \bibitem{gostPZ} ГОСТ 19.404-79. ЕСПД. Пояснительная записка. Требования к содержанию и оформлению. --- М.: ИПК Издательство стандартов, 2001.
        \bibitem{gostRO} ГОСТ 19.505-79. ЕСПД. Руководство оператора. Требования к содержанию и оформлению. --- М.: ИПК Издательство стандартов, 2001.
        \bibitem{gostRP} ГОСТ 19.504-79. ЕСПД. Руководство программиста. Требования к содержанию и оформлению. --- М.: ИПК Издательство стандартов, 2001.
        \bibitem{gostDoc} ГОСТ 19.106-78 Требования к программным документам, выполненным печатным способом. --- М.: ИПК Издательство стандартов, 2001.
  
        \bibitem{gostclimate} ГОСТ 15150-69 Машины, приборы и другие технические изделия. Исполнения для различных климатических районов. Категории, условия эксплуатации, хранения и транспортирования в части воздействия климатических факторов внешней среды. --- М.: Изд-во стандартов, 1997.
    \end{thebibliography} 

    \registrationList
        
\end{document}
