\section{Требования к программе}

\subsection{Требования к функциональным характеристикам}

Функциональным назначением программы является сводный и реляционный анализ данных, экспортируемых из аккаунтов Telegram \cite{telegram}.

\subsubsection{Требования к составу выполняемых функций}

Приложение должно: 

\begin{enumerate}
    \item импортировать данные экспорта сообщений из аккаунта Telegram;
    \item отображать сводную статистику по экспорту;
    \begin{enumerate}
        \item общее количество контактов и сообщений
        \item список известных чатов, каналов для указанного канала
    \end{enumerate}
    \item анализировать связи между выбранными контактами в экспорте;
    \begin{enumerate}
        \item список общих чатов, сообщения в этих чатах
        \item упоминания пользователями друг друга
        \item упоминания общих контактов (A упоминает B, C упоминает B)
        \item упоминания общих чатов/каналов (A, B упоминают один канал)
    \end{enumerate}
\end{enumerate}

\subsubsection{Требования к организации входных данных}

Входные данные - .json файл, экспортированный из приложения Telegram \cite{telegramExport}.

\subsubsection{Требования к организации выходных данных}

Базовая статистика (количество пользователей в экспорте, контактов, базовая информация о аккаунте) выводится в текстовом виде.
Анализ отношений пользователей выводится на экран в виде интерактивного графа, отображающего пользователей и ассоциированные с ними чаты.
В качестве дополнительной информации на графе по разному (тип и цвет объекта) отображаются разные элементы графа: пользователи, личные чаты, публичные и приватные группы, каналы.
С каждым объектом указывается его уникальный Telegram ID.

\subsection{Требования к интерфейсу}

Интерфейс должен позволять пользоваться всеми функциями приложения на полноразмерном (не мене 13in) экране.

\subsection{Требования к надежности}

Приложение не должно аварийно завершаться при любом наборе входных. 
Программа должна обеспечивать проверку корректности входных данных.