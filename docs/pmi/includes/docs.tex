\section{Требования к программной документации}

Состав программной документации:

\begin{enumerate}
    \item ``Визуализация социального графа пользователя Telegram''. Техническое задание \cite{gostTZ}
    \item ``Визуализация социального графа пользователя Telegram''. Программа и методика испытаний \cite{gostPMI}
    \item ``Визуализация социального графа пользователя Telegram''. Текст программы \cite{gostTP}
    \item ``Визуализация социального графа пользователя Telegram''. Пояснительная записка \cite{gostPZ}
    \item ``Визуализация социального графа пользователя Telegram''. Руководство оператора \cite{gostRO}
\end{enumerate}